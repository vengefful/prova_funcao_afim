\documentclass{ufcdocument}

\usepackage[utf8]{inputenc}
\usepackage[portuguese]{babel}

%% Informations that will be insert in the table header 
\def\course{Matemática}
\def\prof{Fernando Jorge}
\def\semester{2022.2}
\def\codeCourse{MA1}
\def\registration{\#ra\#}
\def\student{\#aluno\#}
\def\graduate{\#turma\#}
\def\theme{\#tema\#}

\begin{document}
    %% Table with the header
    \makeheader
    
    %% Space for the instructions
    \fbox{
        \parbox{\textwidth}{
            \begin{minipage}{\textwidth}
                \makeinstructions
                {
                    \begin{instlist}
                        \item Leia atentamente as questões antes de respondê-las.
                        \item É PROIBIDO o uso de qualquer aparelho eletrônico.
                        \item  A avaliação é individual e sem consulta.
                        \item O preenchimento das respostas deve ser feito utilizando caneta (preta ou azul).
                        \item Serão consideradas apenas as respostas que forem acompanhadas de seus cálculos.
                    \end{instlist}
                }
            \end{minipage}
        }
    }
    %% Space between the instructions and the questions.
    \vspace{1cm}
    
    \begin{question}
        \item Sejam os conjuntos $A=\{1,3,5,7,9,11,13\}$ e $B=\{0,2,4,6,8,10,12\}$. Data a função $f:A \rightarrow B$, definida por $f(x) = x - 1$, determine o $D_{(f)}$, o $CD_{(f)}$, a $Im_{(f)}$ e construa o diagrama de flechas que representa essa função. \point{1.0}

        \item Um motorista de táxi cobra, em cada corrida, o valor fixo de R\$ $3,20$ mais R\$ $0,80$ por quilômetro rodado.
        \begin{enumerate}[label=\alph*)]
        	\item Indicando por \textbf{x} o número de quilômetros rodados e por \textbf{P} o preço a pagar pela corrida, escreva a expressão que relaciona \textbf{P} com \textbf{x}. \points{0.30} 
        	\item Determine o número máximo de quilômetros rodados para que, em uma corrida, o preço a ser pago não ultrapasse R\$ $120,00$. \points{0.35}
        	\item Determine qual o valor pago por uma corrida de $17$ quilômetros. \points{0.35}
        \end{enumerate}
    	\item Construa no sistema cartesiano ortogonal o gráfico das funções afins dadas por:
    	\begin{enumerate}[label=\alph*)]
    		\item $f(x) = 2x + 1$ \points{0.30}
    		\item $g(x) = -x + 4$ \points{0.35}
    		\item $h(x) = -2x$ \points{0.35}
    	\end{enumerate}
    	\item Determine a lei da função do $1^\circ$ grau cuja reta passa pelos pontos A(-8,0) e B(0,4). Essa função é crescente ou decrescente? \point{1.0}
    	\item João ao perceber que seu carro apresentara um defeito e optou por alugar um veículo para cumprir seus compromissos de trabalho. A locadora, então, lhe apresentou duas propostas:
    	\begin{itemize}
    		\item \textbf{Plano A}: É cobrado um valor fixo de R\$ 50,00 mais R\$ 1,60 por quilômetro rodado.
    		\item \textbf{Plano B}: É cobrado um valor fixo de R\$ 64,00 mais R\$ 1,20 por quilômetro rodado. 
    	\end{itemize}
    	João observou que, para certo deslocamento \textit{k} quilômetros, era indiferente escolher o \textbf{Plano A} ou \textbf{Plano B}, pois o valor final seria o mesmo. Qual valor de deslocamento em quilômetros é esse? \point{1.0}
    \end{question}

\end{document}